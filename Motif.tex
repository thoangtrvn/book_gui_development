\chapter{Motif (aka OSF/Motif)}
\label{chap:Motif}
\label{chap:osfmotif}

Motif (sometimes called OSF/Motif) is industry-standard to design 2D interface.
We can combine with OpenGL for writing 3D application in which the 2D Window is
created by either OSF/Motif drawing area widgets or OpenGL-specific drawing area
widgets. The 3D rendering context then can be bind to this window.

{\bf motif} is an industry standard toolkit/library for building GUI application
in X windows system since 1980s. Different implementations of the Motif API include Open
Motif, LessTif. 
\begin{itemize}
  \item Open Motif: royalty-free distribution if the software that use Open Motif is open source
  
  \item LessTif: created when Motif still priprietary with the goal to bring Motif-compatible implementation to LGPL licensed
\end{itemize}
From 2012, Motif is LGPLv2 licensed.

Motif was used to
\begin{itemize}
  \item build the GUI for Common Desktop Environment (CDE) - a desktop environment for Unix and OpenVMS O/S
  
  \url{http://en.wikipedia.org/wiki/Common_Desktop_Environment}
  
  
\end{itemize}  


To download motif: \url{motif.ics.com/motif/downloads}

\section{Versions}


Since Motif 2.1, it supports
\begin{itemize}
  \item Unicode
\end{itemize}


\url{http://bwachter.lart.info/documentation/guitoolkits.html}


Motif allows theme creation, but it is not as easy to do as GTK+ (Chap.\ref{chap:GTK+}). One has to
create a long and complex resources file to change the look and feel of Motif
programs. GTK+ users can have this functionality much more easily.

\section{Motif terms}
\label{sec:Motif_terms}

A shell widget acts as the top-level window of the application and handles the
application's interaction with the window manager

Application context
\begin{verbatim}
XtAppContext     app;
\end{verbatim}
is a structure that Xt uses to manage some internal data associated with an
application. Most often, an application receives an opaque pointer to an application context
in the toolkit initialization call and merely passes that pointer to a few other toolkit functions that require it as an argument. 

Even the application context is not being used in most application, the fact
that the application context is a public variable, rather than hidden in the
toolkit internals, is a forward-looking feature of Xt, designed to support
multiple threads of control.

\section{Xm library}
\label{sec:Xm}

The widget set for Motif is implemented in the Xm library
\begin{verbatim}
-lXm
\end{verbatim}
Xm relies on Xt library (Sect.\ref{sec:Xt}).

The header files in \verb!/usr/Motif2.1/include/Xm!
\begin{verbatim}
#include <X11/Intrinsic.h>
#include <X11/Shell.h>
#include <X11/Xatom.h>
#include <Xm/XmStrDefs.h>
#include <Xm/VirtKeys.h>

#include <Xm/PushB.h>  // PushButton widget
\end{verbatim} 
The one with the name ending in a "P" (e.g. PushBP.h) is the widget's private
header file and should not normally be included directly by an application.
Private header files are generally used only by the code that implements a
widget class and its subclasses.

NOTE: \verb!<Xm/XmAll.h>! header file simply includes all of the public
header files.

\subsection{Localization (internalization)}

\begin{verbatim}
XtSetLanguageProc (NULL, NULL, NULL);
\end{verbatim}
\url{http://www.ist.co.uk/motif/books/vol6A/ch-2.fm.html}

\section{How to write a GUI app}

Before an application creates any widgets, it must initialize the toolkit
\begin{itemize}
  \item \verb!XtVaAppInitialize()! (deprecated)
  \item \verb!XtVaOpenApplication()!
\end{itemize}
Example:
\begin{verbatim}
Widget           toplevel;
XtAppContext     app;
toplevel = XtVaOpenApplication (&app, "Hello", NULL, 0, &argc, argv, NULL,
                                sessionShellWidgetClass, NULL);    
\end{verbatim}
which returns a shell widget (Sect.\ref{sec:Motif_terms})


\section{Compile a Motif-based GUI program}

\begin{verbatim}
cc -O -o filename filename.c -lXm -lXt -lX11
\end{verbatim}
The order is important as Xm relies on Xt, and both Xm and Xt rely on Xlib (the -lX11 link flag specifies
Xlib).


\section{Tutorial}

\url{http://www.ist.co.uk/motif/books/vol6A/ch-2.fm.html}