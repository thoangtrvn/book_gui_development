\chapter{Qt}
\label{chap:Qt}

\section{History of Qt}

Developing a cross-platform GUI application was unavailable at that time.
Haavard started developing C++ GUI framework since 1988. In summer 1990, Haavard
Nord and Eirik Chambe-Eng were working together on developing GUI for a C++
database application for ultrasound images. The application should be able to
run with a GUI on different platforms:
Unix, Macintosh and Windows. They came up with an object-oriented display
system, or a object-oriented cross-platform GUI
framework\footnote{\url{http://www.civilnet.cn/book/embedded/GUI/Qt4/pref04.html}}.
In 1991, Haavard started writing classes and collaborated with Eirik on the
design. In 1992, Eirik came up with a simple but powerful GUI programming
paradigm {\bf ``signals and slots''} (Sect.\ref{sec:signal_and_slot}) and
Haavard hand-coded it. In 1993, they completed their Qt's first graphics kernel,
which allows them to build their own widgets. 'Q' was chosen as the classes
prefixes because the letter tool beautiful in Haavard's Emacs font, and 't' =
toolkit (inspired by Xt or X toolkit).

In 1994, the company was first established under the name {\it Quasar
Technologies}, and then renamed as Trolltech; with Haavard Nord (Trolltech's
CEO) and Eirik Chambe-Eng (Trolltech's president). The first public release was
Qt 0.90 on May, 1995. \textcolor{red}{Qt offered the same APIs on both Windows
and Linux}.

KDE project was established in 1996 by Matthias Ettrich who decided to use Qt to
build KDE helped Qt became the de-facto standard in C++ GUI development on
Linux. Matthias joined Trolltech in 1998. Qt 2.0 was released in 1999 with a new
open-source license.

By the end 2000, Trolltech released {\bf Qtopia Core}, a platform for
applications to run on embedded devices with its own windows system as a
replacement for X11. Also, the UNIX version of Qt was made available under GPU
LGPL.

Qt 3.0 was released in 2001, with support for Windows, Mac OS X, Unix and Linux
(desktop and embedded), giving 42 different classes. \textcolor{red}{New
features are Unicode support and improved locale, a Perl-like regular expression
class, new text viewing and editing widget}.

Qt 4.0 was released in 2005, with about \textcolor{red}{500 classes, more than
9000 functions}.
Due to its large size and wide-range of features, it was splitted into different
libraries so that developers need only to link to what they need. The major
changes are: completely new set of efficient and easy-to-use template
containers, advanced model/view, fast and flexible 2D painting framework,
powerful Unicode text viewing and editing classes. For the first time, Qt have
two versions: commercial and open-source development on all platforms that it
supports.

In 2008, Trolltech was then acquired by Nokia with the new name Qt Software, and
Qtopia was renamed to {\bf Qt Extended}, targeting to commercial support.  An
agreement between Nokia and KDE in that Nokia is obliged to continue the
development of Qt and provide it as free
software\footnote{\url{http://www.itworld.com/mobile-wireless/289651/alternate-history-kde}}.
In 2009, Qt Software was renamed to Qt Development Framework; and Qt Extended
was discontinued with latest version 4.4.3. The open-source community took it
over with the new name {\bf Qt Extended Improved}, with latest version 4.5.2 in
June 2009.

Since Qt 4.5, Qt was made available under LGPL version 2.1 license, making it
available for commercial products to use the open-source version of Qt. In 2011,
Qt Development Framework was then sold to Digia PLC.
In Sep 2012, the latest stable release was Qt 4.8.3; and Qt 5.0 beta was
released on Aug, 2012; and Qt 5.0 final to be released on Nov 2012. Qt 5.0
offers full capability of OpenGL/OpenGL ES graphics acceleration.


\section{Qt 5}

On a VNC session, we can test if Qt5 is working by running
\begin{verbatim}
python3

>>> from PyQt5 import QtWidgets
>>> qapp = QtWidgets.QApplication([''])


QXcbConnection: Failed to initialize XRandr
Qt: XKEYBOARD extension not present on the X server.
\end{verbatim}
Xrandr - Sect.\ref{sec:Xrandr}

\subsection{Xrandr}
\label{sec:Xrandr}

\begin{verbatim}
typedef struct {
     int type;               /* event base */
     unsigned long serial;   /* # of last request processed by server */
     Bool send_event;        /* true if this came from a SendEvent request */
     Display *display;              /* Display the event was read from */
     Window window;          /* window which selected for this event */
     int subtype;            /* RRNotify_ subtype */
} XRRNotifyEvent;
\end{verbatim}



\url{https://sourcecodebrowser.com/libxrandr/1.2.3/_xrandr_8h_source.html}

\subsection{EGLFS}
\label{sec:EGLFS}

Since the Qt 5.0 release, Qt no longer contains its own window system (QWS)
implementation. There are multiple platform plugins that are potentially usable
on Embedded Linux systems: EGLFS, LinuxFB, KMS, DirectFB, Wayland. 

EGLFS is a platform plugin for running Qt5 applications on top of EGL and OpenGL
ES 2.0 without an actual windowing system (like X11 or Wayland)

\subsection{Troubleshoot: cross-compile}

\begin{verbatim}
./configure -release -opengl es2 -device linux-rasp-pi2-g++ -device-option
CROSS_COMPILE=$TOOLCHAIN/arm-bcm2708/gcc-linaro-arm-linux-gnueabihf-raspbian/bin/arm-linux-gnueabihf- -sysroot $ROOTFS -prefix /usr/local/qt5
\end{verbatim}


\subsection{Troubleshoot: QXcbConnection, XKEYBOARD}

On Linux, the xcb QPA (Qt Platform Abstraction) platform plugin is used. It
provides the basic functionality needed by Qt GUI and Qt Widgets to run against
X11. It seems that vncserver (which is also a trimmed version of X11
server) doesn't provide many of the X extensions in the free version.

\begin{itemize}
  \item  Qt5.6:\url{http://doc.qt.io/qt-5.6/linux-requirements.html}

  \item Latest Qt 5.9: \url{http://doc.qt.io/qt-5/linux-requirements.html}
\end{itemize}

It's best to compile Qt yourself, using \verb!-qt-xcb! so that it statiscally
link against these X11 extension library instead of trying to link against
the system versions.  This can help make Qt less dependent on some of the xcb
helper libraries that might not be available on all distributions. On Ubuntu,
install these packages
\begin{verbatim}
sudo apt-get install libfontconfig1-dev libfreetype6-dev libx11-dev \
libxext-dev libxfixes-dev libxi-dev libxrender-dev \
libxcb1-dev libx11-xcb-dev libxcb-glx0-dev
\end{verbatim}
Build option \url{http://doc.qt.io/qt-5/configure-options.html}
\begin{verbatim}
./configure --prefix=/packages/qt/5.6 -qt-xcb -qt-zlib -qt-libjpeg -qt-libpng
-qt-xkbcommon -qt-freetype -qt-pcre -qt-harfbuzz
\end{verbatim}


Check the output of the command:
\begin{verbatim}
xprop -root | grep -i xkb

// on Mac
_XKB_RULES_NAMES(STRING) = "base", "empty", "empty", "", ""

// on Linux
CUT_BUFFER0(STRING) = "xprop -root | grep -i xkb "
\end{verbatim}

There are several ways how Qt can obtain a keymap information from the X server,
I would recommend to compile Qt applications with xcb-xkb support,
\begin{verbatim}
1) use xkb_x11_* API (from libxkbcommon)
2) if Qt is compiled with Xlib support then use Xlib + xkbfile calls
3) read _XKB_RULES_NAMES property
\end{verbatim}

Check xrandr
\begin{verbatim}
xrandr -q

 SZ:    Pixels          Physical       Refresh
*0   1800 x 1024   ( 457mm x 260mm )  *60  
Current rotation - normal
Current reflection - none
Rotations possible - normal 
Reflections possible - none
\end{verbatim}

\subsection{Qt 5.6}


\url{https://download.qt.io/archive/qt/5.6/5.6.0/}


\section{signal and slot}
\label{sec:signal_and_slot}

Qt defines an extension to C++ language and thus need to be handled by a special
compiler called {\bf MOC} (Meta-Object Compiler- Sect.\ref{sec:MOC}).

\url{http://qt-project.org/doc/qt-4.8/signalsandslots.html}


Slots are just regular methods. Nothing special there, EXCEPT moc will save
their signature in a table in the intermediate .moc file - you can see this
table quite clearly when you look through this file.

This table allows you to call a method using it's signature.

The SLOT(mySlot(int)) macro boils down to a string representation of the method
in question. There are several ways you can do this, see the documentation for
QMetaObject for example.

When a you connect a signal to a slot, the signal and slot signatures are stored
for later use. When a signal is emitted, all the slots previously connected to
that signal are called using the method described above.

There are 2 types of connection there. First one: slots are called at the time
signal was emitted. And the second one: slots calls are placed in event loop
queue. You can manually select the method in connect, but connecting signal and
slots from different threads are always queued


\section{MOC (Meta-Object Compiler)}
\label{sec:MOC}

Meta-Object Compiler (MOC) handles Qt's C++ extension.
It reads Qt codes and convert the Qt's macros in C++ header file to C++ source
file containing the meta-object code for these classes. The source file
extension is \verb!*.moc!

Qt codes extensively use macros like \verb!Q_OBJECT!.
Qt creates "meta object code" based on "annotations" in your c++ code (eg
\verb!Q_PROPERTY!, \verb!Q_SLOTS! etc).


\url{http://qt-project.org/doc/qt-4.8/moc.html}

\subsection{Meta-object code}

Meta objects enhance programming languages by creating new or manipulate
existing objects. They provide functionalities a language does not actually have
by itself.

The Meta Objects are interpreted either by compile time or run time. In Qt and
C++ it is done during compile time by the Meta Object Compiler (moc) -
Sect.\ref{sec:MOC}). An example case is the usage of the signal/slot concept
(Sect.\ref{sec:signal_and_slot}).

