\chapter{Desktop App Development}

Traditionally, we use widgets, with APIs provided via C/C++ or Java binding to
write GUI Desktop apps. Nowadays, with the advancement of browsers and mobile
devices, it is shifting to writing mobile and web applications (using
JavaScript), that can run as a regular desktop apps. 

The main idea behind developing desktop applications with JavaScript is that you
build one codebase and package it for each operating system separately, i.e.
write once run everywhere (via the webbrowser). Nowadays, developing a desktop
application with JavaScript relies on either Electron or NW.js.

\section{Using Javascript}

NW.js, Electron provides a platform to write desktop applications with
JavaScript and HTML and has Node integration to grant access to the low level
system from web pages. 

\section{-- NW.js (previously: node-webkit)}

Roger Wang, of Intel's Open Source Technology Center, created \verb!node-webkit!
in 2011. This allowed the user to spawn a WebKit browser window and use Node.js
modules within \verb!<script>! tags.

Soon, it was switched from WebKit to Chromium (the open-source project Google
Chrome is based on), and an intern named Cheng Zhao joined the project. It was
soon realized that an app runtime based on Node.js (Sect.\ref{sec:Node.js}) and
Chromium would make a nice framework for building desktop apps.

\begin{mdframed}
node-webkit was later renamed NW.js to make it a bit more generic because it no
longer used Node.js or WebKit.

NW.js is based on io.js (the Node.js fork) at the time, and Chromium had moved
on from WebKit to its own fork, Blink.

\end{mdframed}

\section{-- Electron (previously: Atom Shell)}

Electron was previously named Atom Shell.

