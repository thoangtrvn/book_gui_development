\chapter{GNOME}
\label{sec:GNOME}
\label{chap:GNOME}

GNOME is a desktop environment and GUI that runs on top of an OS. It's free and
an alternate choice to KDE (as KDE use Qt a commercial product). GNOME was
created in 1997, using {\bf GTK+} widget toolkit.  GNOME = GNU Network Object
Model Environment. GTK+ reduces the amount of work when porting one application
to other platforms (Windows, Mac OS X, etc.).


\section{GNOME 1 and GNOME 2}
\label{sec:GNOME-1}
\label{sec:GNOME-2}

GNOME 1 use Nautilus as default file manager (Sect.\ref{sec:nautilus}).
Nautilus is  equivalent to Windows Explorer in Windows.

GNOME 2 use Metacity as default window manager.
Metacity is the Window manager that deals with the behavior of a Window like
maximize, minimize, close, resize, move.

In GNOME 1 and 2, GNOME was designed based on traditional desktop metaphor, i.e.
users can change the looking of the desktop by using different themes (which
usually contains ions, window manager border). 



\section{Cinnamon}
\label{sec:Cinnamon}

Cinnamon is a desktop environment based on GNOME 3 (Sect.\ref{sec:GNOME-3})

As GNOME Shell is far away different from GNOME 2-like user interface, some
efforts have been made to keep GNOME2 layout on top of GNOME Shell. One of that
is {\bf Cinnamon Shell} (Linux Mint distro), which use GNOME Shell (Mutter and
GTK+ 3).

In Cinnamon Shell, window manager is {\bf Muffin}, a fork of Mutter. In Cinnamon
1.6, the default file manager is {\bf Nemo} (not Nautilus).

\section{GNOME 3}
\label{sec:GNOME-3}

Gnome 3 is the third iteration in the Gnome desktop environment. t features a
bar at the top with a clock and a menu button on the left which will bring up a
fullscreen window containing most applications. 

The default theme is rather dark with much black in the general elements, but
the windows are light gray. 


Since GNOME 3, there are many changes that makes it looks similar to Mac OS X
\begin{enumerate}
  \item maximize and minimize button are no longer on the title bar (default)
  
  \item Mutter replaces Metacity as window manager. Mutter is a {\it
  compositing window manager} which allows rendering layout better and faster.
  
  \item GNOME Shell - Sect.\ref{sec:GNOME-Shell} (only works with Mutter)
  replaces GNOME Panel.
  
\end{enumerate}
% Since GNOME 3, it uses {\bf GNOME Shell} (rather than traditional desktop
% metaphor like Nautilus or Metacity) - Sect.\ref{sec:GNOME-Shell}.

\subsection{GNOME Desktop}
\label{sec:GNOME-desktop}

GNOME Desktop on the other hand has lot more than GNOME Shell
(Sect.\ref{sec:GNOME-Shell}). Think of GNOME Desktop as a super set of GNOME
shell; by having also various Gnome specific applications, packages along with
gnome-session.

Example: to install GNOME Desktop, you also need to choose the display manager
(Sect.\ref{sec:Display_manager}), e.g. LightDM - Sect.\ref{sec:LightDM}.

\begin{verbatim}
// OPTION 1
sudo apt-get install ubuntu-gnome-desktop
// and choose lightdm
// during the installation
// Ubuntu comes with LightDM as default display manager

// OPTION 2
// to keep GUI minimal
// will skip extra tools and apps and will install only basic desktop
// environment with few supported tools
sudo apt-get install --no-install-recommends ubuntu-desktop
\end{verbatim}


NOTE: \verb!ubuntu-gnome-desktop! install a full GNOME desktop environment
(including gnome-shell), along with a few standard applications and
optimizations for Ubuntu. Check dependencies
\begin{verbatim}
apt-cache depends ubuntu-gnome-desktop | grep gnome-shell
\end{verbatim}


\subsection{GNOME Shell (2011+)}
\label{sec:GNOME-Shell}

GNOME Shell is simply a desktop environment which changes the user interface.
I would suggest to install GNOME Shell. It will give you a feel of GNOME and you
can easily remove it as well.

GNOME Shell provides a graphical shell and is a more abstract metaphor where
switching between different tasks and virtual workspaces takes place in a
separate area called Overview.

\verb!gnome-shell! will only install the GNOME shell, and its dependencies. In
contrast to \verb!ubuntu-gnome-desktop! (Sect.\ref{sec:GNOME-desktop}), it won't
install the package gnome-session (among others) automatically, which you need
to actually use the GNOME desktop.
\begin{verbatim}
sudo apt-get install gnome-shell

//remove
sudo apt-get remove gnome-shell
\end{verbatim}
GNOME Shell is written in C and JavaScript as a plugin for Mutter.
\url{https://en.wikipedia.org/wiki/GNOME_Shell}


\subsection{Gnome 3.10 with Ubuntu 14.04}
\label{sec:GNOME-3.10}

In an earlier message by Sebastien Bacher's (software engineer at Canonical)
that GTK 3.10 deprecates several options, and thus Ubuntu 14.04 LTS decided to
use Gnome 3.8
\url{http://www.webupd8.org/2013/10/ubuntu-1404-lts-to-stay-on-gtkgnome-38.html}

Later on, however, Ubuntu 14.04 uses GTK 3.10
\url{http://www.webupd8.org/2013/12/gtk-310-lands-in-ubuntu-1404-trusty-tahr.html}
\url{http://www.webupd8.org/2014/01/gnome-shell-310-lands-in-ubuntu-1404.html}

GNOME 3.10 has
\begin{enumerate}
  \item  initial Wayland support, 
  
  \item header bars (client side decorations): it uses the GTK+ toolkit and
  theme engine
  
  \item a new  system menu for GNOME Shell which combines the previous system status menus
  and of course, updated core GNOME apps as well as some new applications.  

  the old system status menus have been replaced with a new System Menu: 

  \item new core apps: 
  Bijiben, a note taking app, GNOME Weather and GNOME photos
  
  \item change in existing core apps:
  
\begin{verbatim}
Web: DuckDuck go is now the default search engine; Web now has a search provider
for GNOME Shell which allows accessing your web history or searching the web
from the GNOME Shell Activities  

Contacts: UI enhancements, updated setup and accounts dialogs

Tweak Tool: redesigned for GNOME 3.10 and includes new options, such as startup
applications management, etc.  

Evince Document Viewer: improved accessibility, new Caret Navigation mode,
improved Djvu support and a new sidebar showing search results  

Clocks: Pretty new analog timer in stopwatch and timer views as well as
geolocation support 

Boxes: support for importing system images (QEMU, raw, vdi - Virtualbox 1.1
image format, vmdk - VMware 3 and 4, vpc and cloop) as well as many UI and other fixes 
\end{verbatim}
  
\end{enumerate}
\url{http://www.webupd8.org/2013/09/see-whats-new-in-gnome-310-video.html}

\subsection{Gnome 3.12+ issues with Ubuntu 14.04}
\label{sec:GNOME-3.12}

Ubuntu 14.04 uses Gnome 3.10.  However, after Gnome 3.10, e.g. Gnome 3.12+,
there happened huge differences on Gnome side, it's not compatible with current
Ubuntu. Even compiz won't work properly. Ubuntu team patches Gnome to use it.
\url{https://askubuntu.com/questions/497955/how-do-you-update-gtk-and-or-gnome-on-ubuntu-14-04-lts}

If you want to upgrade, follow the steps
\url{https://wiki.ubuntu.com/UbuntuGNOME/HowTo/UpgradeGnomeShell}

\url{https://launchpad.net/~gnome3-team/+archive/ubuntu/gnome3-staging/+index?batch=75&direction=backwards&memo=225&start=150}
\begin{verbatim}
// GNOME 3.12 in Ubuntu 14.04

//remove first
ppa-purge gnome3-team/gnome3-staging
ppa-purge gnome3-team/gnome3

// then
sudo add-apt-repository ppa:gnome3-team/gnome3-staging
sudo add-apt-repository ppa:gnome3-team/gnome3
sudo apt-get update

  // if already have GNOME
  // also choose a Display Manager, e.g. GDM or LightDM
sudo apt dist-upgrade
 // if not, install GNOME
sudo apt-get install gnome gnome-shell
\end{verbatim}

If you're going GNOME-only then GDM is fine. But if you plan to switch between
GNOME and other DEs I recommend sticking with LightDM.


GNOME 3.12 has
\begin{enumerate}
  \item full Wayland support - Sect.\ref{sec:Wayland},
  
  \item introduce a Sound Recorder preview release, 
  
  \item Systemd for the user session, 
  
  \item colour tinting in GNOME Shell, 
  
  \item videos application implementation, support for Facebook in GNOME photos,
  integrate Zimbra in GNOME, Git integration in the developer experience as well
  as ratings, screenshots and history support in GNOME Software.
\end{enumerate}


\chapter{Ubuntu}

Ubuntu is based on Debian distro (Sect.\ref{sec:Debian}).

\section{Debian vs. Ubuntu}
\label{sec:Debian}

Check Debian version in Ubuntu
\begin{verbatim}
cat /etc/debian_version
\end{verbatim}
\url{https://askubuntu.com/questions/445487/what-debian-version-are-the-different-ubuntu-versions-based-on}

\subsection{Debian vs. Ubuntu versions}
\label{sec:Debian-versions}

The following lines give a mapping from Ubuntu version to Debian version.

sid is the development distribution of Debian (sid - testing - stable)
\begin{verbatim}
18.04  bionic     buster  / sid
17.10  artful     stretch / sid
17.04  zesty      stretch / sid
16.10  yakkety    stretch / sid
16.04  xenial     stretch / sid
15.10  wily       jessie  / sid
15.04  vivid      jessie  / sid
14.10  utopic     jessie  / sid
14.04  trusty     jessie  / sid
13.10  saucy      wheezy  / sid
13.04  raring     wheezy  / sid
12.10  quantal    wheezy  / sid
12.04  precise    wheezy  / sid
11.10  oneiric    wheezy  / sid
11.04  natty      squeeze / sid
10.10  maverick   squeeze / sid
10.04  lucid      squeeze / sid
\end{verbatim}

Ubuntu is often ahead of Debian on core packages like libc6. Trying to install a
package built on Ubuntu on a contemporary version of Debian is likely to end up
with version errors on libc6.

% Debian version on which your Ubuntu version is based in the file:
% \verb!/etc/debian_version!

% \begin{verbatim}
% 17.10  artful     stretch / sid
% 17.04  zesty      stretch / sid
% 16.10  yakkety    stretch / sid
% 16.04  xenial     stretch / sid
% 15.10  wily       jessie  / sid
% 15.04  vivid      jessie  / sid
% 14.10  utopic     jessie  / sid
% 14.04  trusty     jessie  / sid
% 13.10  saucy      wheezy  / sid
% 13.04  raring     wheezy  / sid
% 12.10  quantal    wheezy  / sid
% 12.04  precise    wheezy  / sid
% 11.10  oneiric    wheezy  / sid
% 11.04  natty      squeeze / sid
% 10.10  maverick   squeeze / sid
% 10.04  lucid      squeeze / sid
% \end{verbatim}
% NOTE: sid is the development distribution of Debian (sid - testing - stable)

% Debian version
% \begin{verbatim}
% Debian 9 (stretch)
% 
% Debian 8 (jessie) 
% 
% Debian 7 (wheezy) 
% \end{verbatim}
\url{https://www.debian.org/releases/}

\subsection{Debian 9 (stretch)}
\label{sec:Debian-stretch}

Uses
\begin{verbatim}
Linux kernel 4.9 LTS
apt 1.4.6
gcc 6.3

libc version 2.24-11

Gnome 3.22 
KDE Plasma 5.8 LTS
Xfce 4.12

Database: MariaDB 10.1 (a MySQL variant)
      to replace MySQL 5.5 or 5.6

systemd version 232  
\end{verbatim}

\url{https://www.debian.org/News/2017/20170617}
\section{debuild vs. dpkg-buildpackage}
\label{sec:dkpg-buildpackage}

debuild and dpkg-buildpackage initialize the environment variable  CFLAGS to a
certain value.

\subsection{clean code for clean build}

As the \verb!debian/rules! file actually calls a Makefile, then this Makefile
should have a \verb!clean! target:
\begin{verbatim}
make clean
\end{verbatim}

or
\begin{verbatim}
dpkg-buildpackage -rfakeroot -Tclean
\end{verbatim}

\subsection{debuild}
\label{sec:debuild}

debuild is a convenient wrapper around dpkg-buildpackage, fakeroot, lintian, and
debsign. It handles all the packaging linting, and signing for us.

debuild will make use of symlinks in it's build process, so if you are using a
Virtualbox or VMware Shared folder to build your package, it will not work.

NOTE: debuild and dpkg-buildpackage initialize the environment variable  CFLAGS to a
certain value.

REQUIREMENTS:
\begin{enumerate}
  \item a \verb!debian/! folder
  
  where all of our debian specific packaging files will go.
  
  \item a \verb!debian/control! file
  
  Example: it describes your package using debian control fields. 
  \begin{verbatim}
  Source: helloworld  
Maintainer: Julio Capote <julio@packagecloud.io>
Build-Depends: debhelper (>= 8.0.0)
Standards-Version: 3.9.3
Section: utils

Package: helloworld  
Priority: extra  
Architecture: any
Depends: ${shlibs:Depends}, ${misc:Depends}  
Description: a simple helloworld package
 Just prints "Hi", it's very useful.
  \end{verbatim}
  
  \item a \verb!debian/changelog! file
  
  This file tracks all the changes to the package. You need a minimum of one
  entry to build a package.
  
  The syntax of this file is tricky, and it is recommended to use
  \verb!debchange! utility (or dch) to create this file.
  \begin{verbatim}
  dch -i -create
  \end{verbatim}
  
  Make sure you change UNRELEASED to unstable
  
  \item a \verb!debian/rules! file
  
  This file tells debian specifically how to build your package. The most basic
    implementation just passes all calls to the original Makefile
   
\begin{verbatim}
#!/usr/bin/make -f  
%:  
	dh $@
\end{verbatim} 
    
    \item a \verb!debian/copyright! file
    
 It can be empty
 \begin{verbatim}
 Copyright 2015, Computology LLC.
 \end{verbatim}
\end{enumerate}


