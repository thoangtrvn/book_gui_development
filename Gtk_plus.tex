\chapter{GTK+}
\label{chap:GTK+}

\section{GIMP (GTK)}
\label{sec:GTK}

GIMP is released under LGPLv3 and GPLv3+. GIMP is an open-source raster image
editor: image retouching, editing, free-form drawing, resizing, cropping,
photo-montages, converting between different image formats, etc.

The GUI tookit for GIMP is called GTK ({\bf GIMP ToolKit}), which was developed
in parallel to facilitate the development of GIMP software.

Since GIMP 0.6, GTK was designed to replace Motif GUI toolkit
(Chap.\ref{chap:Motif}), i.e. giving prettier look. Since GIMP 0.99, GTK+ was
used (Sect.\ref{sec:GTK+}).

Applications that use GTK: OpenOffice, Firefox, Gaim, GIMP.

There are different wrappers
\begin{enumerate}
  \item GTKmm - Sect.\ref{sec:GTKmm}
  \item GTK+ - Sect.\ref{sec:GTK+}
  \item GTK\# - Sect.\ref{chap:GTKsharp}
\end{enumerate}

\section{GTKmm}
\label{sec:GTKmm}

GTKmm is the C++ wrapper for GTK (Sect.\ref{sec:GTK}).

\section{GTK+}
\label{sec:GTK+}

GTK+ was designed to replace GTK (Sect.\ref{sec:GTK}) in GIMP 0.99.
Cario is a 2D graphic library (Sect.\ref{sec:cairo}), and GTK+ uses it to render
widgets. Cario is low-level.

In the future, it's entirely possible that GTK+ will use Clutter
(Sect.\ref{sec:Clutter}) internally as the base for its widgets - though that's
still a work in progress.
\begin{verbatim}
  GPU <- [ [ Cogl + Cairo ] <- [ GDK + Clutter ] <- GTK+ ] <- application
\end{verbatim}
\url{https://lwn.net/Articles/608501/}

\begin{mdframed}

If you are an app developer you might use GTK+ to create the window, menus,
toolbars, etc and use clutter to draw any app content that is "drawing area
like" and doesn't fit into the widget-based app metaphor. Within Clutter, you
may dig down to Cairo for specific drawing tasks.

\end{mdframed}

GTK+ was designed using object-oriented programming approach based on the
base object called GObject (Sect.\ref{sec:GObject}).

Compared to Motif, GTK+ has a much more advanced object system than Motif. This
is a strong point for GTK+ that make it easier to program.
Another advantage of GTK+ compared to Motif is that GTK+ can easily change
themes. That is, one can devise different look and feels for a GTK+ program by
just changing simple configurations.

GTK+ integrates better with other libraries written for UNIX. Due to its modular
nature, GTK+ can be updated more easily.
\url{http://in4mationflow.com/Prog/Gtk/Gtk-in-Comparison-with-Motif} 

\subsection{check version}

\begin{verbatim}
dpkg -l libgtk* | grep -e '^i' | grep -e 'libgtk-*[0-9]'

apt-cache policy libgtk2.0-0 libgtk-3-0 
\end{verbatim}
list all the libgtk packages, including -dev ones, that are on your system.

Check GTK+ 1.x
\begin{verbatim}
pkg-config --modversion gtk+
\end{verbatim}


Check GTK+ 2.x
\begin{verbatim}
pkg-config --modversion gtk+-2.0
\end{verbatim}

Check GTK+ 3.x
\begin{verbatim}
pkg-config --modversion gtk+-3.0
\end{verbatim}
\url{http://stackoverflow.com/questions/126141/how-do-you-find-out-which-version-of-gtk-is-installed-on-ubuntu}



\subsection{install}

You can run GTK+ apps on KDE (just like you can run Qt apps under Gnome) as long
as you have the appropriate libraries installed. Gnome is the desktop
"environment" and a development platform, individual applications can use
whatever toolkit they like (though it's true that using GTK+ makes interaction
with other Gnome applications easier).

However, some GTK+ apps use components/utilities from the GNOME desktop
environment. To get your GTK+ app to run, you may have to install components
from GNOME. It is recommended to install GNOME
Shell (Sect.\ref{sec:GNOME-Shell}).
\url{https://stackoverflow.com/questions/3339828/is-it-possible-to-run-gtk-apps-without-gnome-or-the-likes}


GTK+3.12 comes with Gnome3.12 (Sect.\ref{sec:GNOME-3.12}). To install GTK+3.12
on Ubuntu 14.04
\url{http://www.webupd8.org/2014/05/how-to-install-gnome-312-in-ubuntu.html}


To install the latest version of GTK+
\begin{verbatim}
sudo add-apt-repository ppa:gnome3-team/gnome3
sudo add-apt-repository ppa:gnome3-team/gnome3-staging
\end{verbatim}

 \subsection{GTK+ 2.0}
 
 \subsection{-- GTK+ 2.8}
 \label{sec:GTK+-2.8}
 
GTK+ 2.8 (2005) began the transition to using Cairo (Sect.\ref{sec:cairo}) to
render the majority of its graphical control elements.
 
 \subsection{GTK+ 3.0}

Since GTK+ 3.0 (2005), all graphical control elements are rendered using
Cairo (Sect.\ref{sec:cairo}).
 
GTK+ 3.13.3 contains 203 active and 37 deprecated widgets.

\subsection{Popover widget}
\label{sec:Popover-widget}

Popover was a new widget added in 3.12, which is what gotk3 currently targets
with the default build

