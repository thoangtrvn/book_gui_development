\chapter{WPF (Windows Presentation Foundation)}
\label{chap:WPF}

\section{Introduction}

Typically, writing a GUI app requires at least two threads, one for handling the
main GUI to prevent inresponsiveness for time-consuming task. This requires the
programmers to deal with threading explicitly.

Typically, WPF applications start with two threads: one for handling rendering
and another for managing the UI.
\begin{enumerate}
  \item {\bf rendering thread}: effectively runs hidden in the background -
  Sect.\ref{sec:rendering-thread}
  
  \item {\bf UI thread}: receives input, handles events, paints the screen, and
  runs application code - Sect.\ref{sec:UI-thread} 

Most WPF-based applications use a single UI thread, but there are still
situations where multiple threads improve user interface (UI) responsiveness or
application performance.


\end{enumerate} 

WPF is the latest Microsoft's approach to GUI framework to be used in .NET
framework. A {\bf GUI framework} provides a framework for programmers to create
an application with a wide range of GUI elements, e.g. label, textboxes, \ldots.
This is considered a newer option compared to WinForms
(Chap.\ref{chap:Windows.Forms}).

WPF use XAML which makes it easy to create and edit the GUI
(Sect.\ref{sec:XAML}), while the code for handling the GUI's event can be
written in any .NET languages. To write UI using WPF, we can use
\begin{itemize}
  \item Microsoft Visual Studio (good for both coding and XAML editing)
  \item Microsoft Expression Bend (for designers, part of Expression Studio),
  with supports for gradients, template editing, animation, etc.
  
  Expression Bend can open solution files created by Microsoft VS
\end{itemize}

\url{https://msdn.microsoft.com/en-us/library/ms741870(v=vs.100).aspx}

\subsection{UI thread}
\label{sec:UI-thread}

There is an object of class {\bf Dispatcher} that provides services for managing
the queue of work items for a thread.
Every UI thread must have at least one Dispatcher, and each Dispatcher can
execute work items in exactly one thread (Sect.\ref{sec:Dispatcher_class}), it
selects on a priority basis and runs each one to completion. 

The trick to building responsive, user-friendly applications is to maximize the
Dispatcher throughput by keeping the work items small, i.e. not taking much
time to complete.  Any perceivable delay between input and response can
frustrate a user.



\subsection{(background) rendering thread}
\label{sec:rendering-thread}

Historically, Windows allows UI elements to be accessed only by the thread that
created them. This means that a background thread in charge of some long-running
task cannot update a text box when it is finished. Windows does this to ensure
the integrity of UI components. WPF has a built-in mutual exclusion mechanism
that enforces this coordination. 

If only one thread can modify the UI, how do background threads interact with
the user? - A background thread can ask the UI thread to perform an operation on
its behalf. It does this by registering a work item with the Dispatcher of the
UI thread. The Dispatcher class provides two methods for registering work items: Invoke and BeginInvoke.
Both methods schedule a delegate for execution (NOTE: a delegate is similar to
a function pointer in C/C++, but it is type-safe, i.e. can be checked by the
compiler - check C\# book on how to create a delegate)
  
\begin{itemize}
  \item Dispatcher.Invoke (delegate): synchronous call - doesn't
  return until the UI thread actually finishes executing the delegate

\begin{lstlisting}
    // Place delegate on the Dispatcher.
    this.Dispatcher.Invoke(DispatcherPriority.Normal,
        new TimerDispatcherDelegate(TimerWorkItem));
\end{lstlisting}    

  \item Dispatcher.BeginInvoke: asynchronous call
\end{itemize}

\url{https://msdn.microsoft.com/en-us/library/system.windows.threading.dispatcher.invoke(v=vs.100).aspx}

\subsection{DispatcherObject}
\label{sec:DispatcherObject_class}

Most classes in WPF derive from DispatcherObject.
At construction, a DispatcherObject object stores a reference to the Dispatcher
object linked to the currently running thread. So

\url{https://msdn.microsoft.com/en-us/library/system.windows.threading.dispatcherobject(v=vs.100).aspx}

\subsection{Dispatcher}
\label{sec:Dispatcher_class}

Once an object Dispatcher is created in a thread, it maintains a prioritized
queue of work item for a specific thread. 

If you attempt to get the CurrentDispatcher for the current thread and a
Dispatcher is not associated with the thread, a Dispatcher will be created. A
Dispatcher is also created when you create a DispatcherObject. If you create a
Dispatcher on a background thread, be sure to shut down the dispatcher before
exiting the thread.

\url{https://msdn.microsoft.com/en-us/library/system.windows.threading.dispatcher(v=vs.100).aspx}

\section{WPF with Visual C++}

To create the user interface in WPF we use Extensible Application Markup
Language (XAML - Sect.\ref{sec:XAML}. It's primary measuring unit is not pixel
based, so applications display at any dpi (Resolution independence using vector graphics).
WPF is using DirectX that can enjoy the benefits of hardware acceleration for
smoother graphics and overall better performance.

\url{http://www.bogotobogo.com/CSharp/csharp_wpf_xaml_netframework.php}

\section{XAML}
\label{sec:XAML}


XAML is XML-based language, whose file is used to describe the GUI structures.
XAML is for designer and thus can be split from the
programming part (C\#, VB.NET, \ldots).
\begin{itemize}
  \item use hardware acceleration for drawing the GUI: better performance
  \item UI works for both Desktop-based and Web-based (using Silverlight/XBAP)
  \item first choice by Microsoft, which is being used in newer Visual Studio
  tools
\end{itemize}

All measures in WPF is based on logical units, not pixels. A logical unit = 1/96
of an inch. So, the size of each components doesn't change when you change the
resolution of your screen. However, you can also easily build a scalable UI
using the vector-based rendering engine.

The UI is separated from the behavior (i.e. how each component reacts when user
clicks/selects). The UI is implemented using XAML (Sect.\ref{sec:XAML}); while
the behavior is implementd using C\# or Visual Basic.

In WPF, we can virtually define any types of controls as
content of another.

Example: put an image into a button to create an image button
\begin{verbatim}
<Button>
    <StackPanel Orientation="Horizontal">
        <Image Source="speaker.png" Stretch="Uniform"/>
        <TextBlock Text="Play Sound" />
    </StackPanel>
</Button>
\end{verbatim}

Example: a simple GUI, with a button and a delegate \verb!StartOrstop()! to
handle mouse-click on the button
\begin{verbatim}
<Window x:Class="SDKSamples.Window1"
    xmlns="http://schemas.microsoft.com/winfx/2006/xaml/presentation"
    xmlns:x="http://schemas.microsoft.com/winfx/2006/xaml"
    Title="Prime Numbers" Width="260" Height="75"
    >
  <StackPanel Orientation="Horizontal" VerticalAlignment="Center" >
    <Button Content="Start"  
            Click="StartOrStop"
            Name="startStopButton"
            Margin="5,0,5,0"
            />
    <TextBlock Margin="10,5,0,0">Biggest Prime Found:</TextBlock>
    <TextBlock Name="bigPrime" Margin="4,5,0,0">3</TextBlock>
  </StackPanel>
</Window>

<Window x:Class="SDKSamples.MainWindow"
    xmlns="http://schemas.microsoft.com/winfx/2006/xaml/presentation"
    xmlns:x="http://schemas.microsoft.com/winfx/2006/xaml"
    Title="Prime Numbers" Width="260" Height="75"
    >
    <StackPanel Orientation="Horizontal" VerticalAlignment="Center" >
        <Button Content="Start"  
            Click="StartOrStop"
            Name="startStopButton"
            Margin="5,0,5,0"
            />
        <TextBlock Margin="10,5,0,0">Biggest Prime Found:</TextBlock>
        <TextBlock Name="bigPrime" Margin="4,5,0,0">3</TextBlock>
    </StackPanel>
</Window>
\end{verbatim}

then in the code, we need to define \verb!Window1! class in the
\verb!SDKSamples! namespace
\begin{lstlisting}
namespace SDKSamples
{
    public partial class Window1 : Window
    {
    	public Window1() : base()
        {
            InitializeComponent();
        }
		private void StartOrStop(object sender, EventArgs e)
        { // Delegate
        }
    }
    
}
\end{lstlisting}
\url{https://msdn.microsoft.com/en-us/library/ms741870(v=vs.100).aspx}

\subsection{Delegate}
\label{sec:Delegate}

