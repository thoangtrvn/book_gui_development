\chapter{Windows.Forms}
\label{chap:Windows.Forms}

Platforms: Windows, Unix, OSX.

Microsoft's \verb!System.Windows.Forms! (aka Managed.Windows.Forms, MWF,
Winforms) is a binding developed by Microsoft to the Win32 toolkit. It is 
 a popular toolkit used by millions of Windows developers (especially for
internal enterprise applications).

The Mono project decided to produce a compatible implementation (Winforms) to
allow these developers to easily port their applications to run on Linux and other Mono platforms.   
Mono's System.Windows.Forms ({\bf MWF}) is implemented using
\verb!System.Drawing!.
All controls are natively drawn through System.Drawing.
System.Windows.Forms implements its own driver interface to communicate with the
host OS windowing system. The available driver interfaces supports X11, Win32,
OSX.

Most of the Windows.Forms API will work on Mono, however some applications (and
especially third party controls) occasionally bypass the API and P/Invoke
straight to the Win32 API. These calls will likely have to changed to work on
Mono.  

In general, Winforms applications are written using Microsoft's Visual Studio or
SharpDevelop, which both provide a visual designer for creating Winforms GUIs.
As WinForms is designed for Desktop-based applications; many people are shifting
to using WPF which is designed to be both Desktop-based and Web-based friendly
(Chap.\ref{chap:WPF}). However, at the moment, WinForms has more supported GUI
elements; while with WPF you may need to buy third-party packages.

\section{Windows Form with C++/CLI}

Before Visual Studio 2012
\begin{verbatim}
File/New/Project...
    Visual C++ (project types)
       CLR
          Windows Forms Application (Templates)
\end{verbatim}

According to the VC++ team, the primary intent of C++/CLI was to provide for
great interop solutions, and not another full-featured .NET language. So if
you're looking to build a new .NET based desktop apps, C\# or VB.NET are the
recommended choices going forward.  So, since Visual Studio 2012, this feature
is removed from the Wizzard.
\url{https://social.msdn.microsoft.com/Forums/vstudio/en-US/2acadc0f-33b6-43e2-aff7-52654cc48ab8/how-to-create-windows-forms-application-in-vs-c-2013-rc?forum=vcgeneral}

However, we still can create a Winforms application using C++/CLI, from Visual
Studio 2012
\begin{verbatim}
File/New/Project...
    Visual C++ (project types)
       CLR
          CLR Empty Project	
\end{verbatim}
Suppose you name the project is CUDAFLuxGUI.

Then Project/Add New Item\ldots
\begin{verbatim}
Visual C++
   UI
     Windows Form
\end{verbatim}
and create a new form name (e.g. MainForm.h), then edit \verb!MainForm.cpp!. In
the code (NOTE: You can also copy the form designed in C\# project; however
partial class is not supported here).

\begin{verbatim}
#include "MainForm.h"

using namespace System;
using namespace System::Windows::Forms;

[STAThread]
void main(array<String^>^ arg) 
{
   Application::EnableVisualStyles();
   Application::SetCompatibleTextRenderingDefault(false);
   
   //Create the form instance
   CUDAFluxGUI::MainForm form;
   
   Application::Run(%form);

}
\end{verbatim}
The System namespace provides functions to work with UI controls. Then,
right-click on the project, get \verb!Properties! window
\begin{verbatim}
Configuration Properties
    Linker
        System
\end{verbatim}
select  Windows (/SUBSYSTEM:WINDOWS) for SubSystem. If you don't see
\verb!SubSystem!, make sure you change the configuration setting from Win32 to
x64. (See Sect.\ref{sec:configuration-properties_Linker_subsystem} in
Visual C++ chapter in C/C++ manual book)

\begin{verbatim}
Advanced
   Entry point
\end{verbatim}
type in \verb!Main!; then hit OK button. 

IMPORTANT: Make sure the above settings are set for both Debug and Release
configuration.

Finally, hit F5 to test the program.

\url{http://www.bogotobogo.com/cplusplus/application_visual_studio_2013.php}


If you check the \verb!MainForm.h! file, it has
\begin{verbatim}
#pragma once

namespace CUDAFluxGUI {

	using namespace System;
	using namespace System::ComponentModel;
	using namespace System::Collections;
	using namespace System::Windows::Forms;
	using namespace System::Data;
	using namespace System::Drawing;

	/// 

	/// Summary for MyForm
	/// 

	public ref class MainForm : public System::Windows::Forms::Form
	{
	public:
		MainForm(void)
		{
			InitializeComponent();
			//
			//TODO: Add the constructor code here
			//
		}
        ...

\end{verbatim}

Explain:
\begin{itemize}
  \item \verb!#pragma once! : it means only open this file once during the
  compilation. \verb!System! namespace gives us functions to deal with UI
  controls.
  \item a class name \verb!MainFrom! is derived from the Windows Form
  \verb!Form! base class.
\end{itemize}



To deploy the system, create another project 
\begin{verbatim}
Installed
   InstallShielf Limit ... Setup and Deployment
\end{verbatim}
and name it \verb!Setup_CUDAFluxGUI!. From the Project Assistant window, we
setup the properties of the installation. 

\section{Windows Form with .NET language (C\#)}

\section{Form class}
\label{sec:Form-class}

\begin{verbatim}
System.Windows.Forms.Form 
\end{verbatim}
Represents a window or dialog box that makes up an application's user interface.
The main windows of an application is a derivative of this class.

\url{http://msdn.microsoft.com/en-us/library/system.windows.forms.form(v=vs.110).aspx}

\begin{verbatim}
namespace CUDAFluxGUI
{
    public partial class MainForm : Form
    {
    	// (private) data members
    	
    	// constructor
    	public MainForm() : base()
    	{
    	  // do GUI setting
    	  // initialize data members
    	}
    }
}
\end{verbatim}

You have the option to create MDI application (multiple-document interface),
that allows you to open multiple files, and a Windows menu to switching from one
document to another.
\url{http://msdn.microsoft.com/en-us/library/xyhh2e7e(v=vs.110).aspx}

\section{Application class (sealed)}
\label{sec:Winforms.Application}

\begin{verbatim}
System.Windows.Forms.Application
\end{verbatim}
class is an important class, it contains \verb!static! method that can be used
to control the GUI of the application and other tasks (e.g. start/stop the
application).

Example: below is the core functional calls in the Main() method
\begin{verbatim}
using System.Windows.Forms;

public class Form1 : System.Windows.Forms.Form
{

[STAThread]
static void Main() 
{
// the first line in Main() method
   Application.EnableVisualStyles();

// to use GDI based TextRenderer class for text rendering
   Applications.SetCompatibleTextRenderingDefault(false);

// Add the event handler for handling UI thread exceptions to the event.
   Application.ThreadException += new
      ThreadExceptionEventHandler(ErrorHandlerForm.Form1_UIThreadException);
    
    
   Application.Run(new Form1());
}

public Form1()
{
 //UI components here
}
\end{verbatim}
The \verb!Run()! method start the application message loop on the current
thread, and (optionally) make the form visible.
\begin{verbatim}
Application.Run() 	// without a form

Application.Run(new Form1()) // make the form visible
\end{verbatim}

\url{http://msdn.microsoft.com/en-us/library/system.windows.forms.application(v=vs.110).aspx}


\section{Colors}


\begin{verbatim}
System.Drawing.Color class

System.Drawing.SystemColors class
\end{verbatim}

Example:
\begin{verbatim}
Color toolStripMenuItemSelectedColor = Color.FromArgb(194, 224, 255);
Color toolStripMenuItemUnSelectedColor = SystemColors.Control;
\end{verbatim}

\section{Menu}

\begin{verbatim}
System.Windows.Forms.Menu
  System.Windows.Forms.MainMenu  //the class is replaced by 
                                   MenuStrip  (.NET 2.0+)
  
System.Windows.Forms.Menu
  System.Windows.Forms.ContextMenu  // the class is replaced by
                                      ContextMenuStrip (.NET 2.0+)
        
  
System.Windows.Forms.Control
  System.Windows.Forms.ScrollableControl
    System.Windows.Forms.ToolStrip
      System.Windows.Forms.MenuStrip   // represent a menu system for a
                                        Form (or it derivative class)
            

System.Windows.Forms.Control
  System.Windows.Forms.ScrollableControl
    System.Windows.Forms.ToolStrip
      System.Windows.Forms.ToolStripDropDown
        System.Windows.Forms.ToolStripDropDownMenu
          System.Windows.Forms.ContextMenuStrip   // represent a shortcut menu                                   
\end{verbatim}

\section{Menu Item}

\begin{verbatim}
System.Windows.Forms.Menu
  System.Windows.Forms.MenuItem  // a single menu item displayed on
                            MainMenu or ContextMenu 
                           the class is replaced by ToolStripMenuItem


      System.Windows.Forms.ToolStripItem
        System.Windows.Forms.ToolStripDropDownItem
          System.Windows.Forms.ToolStripMenuItem  // a selectable menu item on
                                                    MenuStrip or
                                                    ContextMenuStrip
\end{verbatim}

\section{SplitContainer}

SplitContainer class represents a UI control with two sides, and a movable
splitter, which can be configured horizontal spliter or vertical splitter.


\section{Helloworld app}


The programs has the main file (MainForm.cs) with the minimal included headers
\begin{Verbatim}
using System;
using System.Collections.Generic;
using System.ComponentModel;
using System.Data;
using System.Drawing;
using System.Linq;
using System.Text;
using System.Windows.Forms;
\end{Verbatim}

\begin{mdframed}
This is in Mono

\begin{verbatim}
using System;
using System.Drawing;
using System.Windows.Forms;
\end{verbatim}

\end{mdframed}

Next step: put the main class inside a namespace. The name of the namespace is
typically the name of the application
\begin{Verbatim}
namespace CUDAFluxGUI
{
   //put the main GUI class here (e.g. CUDAFluxGUI class)

}
\end{Verbatim}

The main class is defined here
\begin{enumerate}
  \item it is a partial class of the \verb!Form! class
  (Sect.\ref{sec:Form-class})
  \item A public constructor also need to call the default constructor of
  \verb!Form! class.
  
\begin{Verbatim}
public MainForm()
     : base()
 {
\end{Verbatim}
   and call automatic methods that should not be modified
\begin{verbatim}
     InitializeComponent();
     WindowState = FormWindowState.Maximized;     
}
\end{verbatim}


\end{enumerate}

\begin{Verbatim}
//public partial class HelloWorld : Form
public partial class MainForm : Form
    {
        ///

        /// Main constructor for HelloWorld Form
        /// 


        public HelloWorld()
        {
            // initialize the designer standard components
            InitializeComponent();

            // make sure that this window is maximized
            WindowState = FormWindowState.Maximized;
        }

        ///

        /// Called when the form is first loaded.  Resize the HelloWorldLabel to fit the new form size
        /// 


        ///
Ignored
        ///
Ignored
        private void HelloWorld_Load(object sender, EventArgs e)
        {
            ResizeHelloWorld();
        }

        ///

        /// Called when the form is resized.  Resize the HelloWorldLabel to fit the new form size
        /// 


        ///

        ///

        private void HelloWorld_Resize(object sender, EventArgs e)
        {
            ResizeHelloWorld();
        }

        ///

        /// Resize the Hello World to fill the client area
        /// 


        void ResizeHelloWorld()
        {
            // change the font to match the size of the client area
            HelloWorldLabel.Font = GetBestFontFit(ClientSize, HelloWorldLabel.Font, HelloWorldLabel.Text, ClientSize.Height, 8f); ;

            // center the label
            CenterHelloWorld();
        }

        ///

        /// Center the HelloWorldLabel in the client area of HelloWorld form
        /// 


        void CenterHelloWorld()
        {
            HelloWorldLabel.Location = new Point((ClientSize.Width / 2) - (HelloWorldLabel.Width / 2), (ClientSize.Height / 2) - (HelloWorldLabel.Height / 2));
        }

        ///

        /// Determine the best font match for the size/text.  The result should be between the maximum and minimum height.
        /// The size is the area to fill.  The new font is based on the old font which is passed in.
        /// 


        /// Font
        public static Font GetBestFontFit(Size size, Font currFont, String text, float fontMax, float fontMin)
        {
            float fontPixels = fontMax;
            float minfontPixels = fontMin;

            // find the corresponding font size for the textbox size
            Font font = currFont;

            // make sure that the text is not null or empty before attempting to fit it in the area.
            // if it is null or empty, the current font is used since it does not matter.
            if (!String.IsNullOrEmpty(text))
            {
                Size textsize;

                font = new Font(currFont.FontFamily, fontPixels, currFont.Style, GraphicsUnit.Pixel);

                textsize = TextRenderer.MeasureText(text, font);

                // Check to see if the font fits in the area and if not, find a smaller one
                while ((((textsize.Height) > size.Height) || ((textsize.Width) > size.Width)) && (fontPixels > minfontPixels))
                {
                    // try to speed up the process of finding the right font size by detecting how much it is too big by
                    if (textsize.Height > size.Height)
                    {
                        // reduce the fontPixels based on how far off the font size is
                        fontPixels -= (textsize.Height - size.Height);
                    }
                    else if (textsize.Width > size.Width)
                    {
                        int charPixelWidth = textsize.Width / text.Length;
                        int targetPixelWidth = size.Width / text.Length;

                        // if the characters are too wide, reduce the fontPixels by how different the result is from the desired width
                        if (charPixelWidth > targetPixelWidth)
                        {
                            fontPixels -= (charPixelWidth - targetPixelWidth);
                        }
                        else
                        {
                            fontPixels -= 1.0F;
                        }
                    }
                    else
                    {
                        fontPixels -= 1.0F;
                    }

                    // drop the old one since it did not work
                    font.Dispose();

                    // get a new font based on the smaller font pixels size
                    font = new Font(font.FontFamily, fontPixels, font.Style, GraphicsUnit.Pixel);

                    // recalculate the size based on the new font
                    textsize = TextRenderer.MeasureText(text, font);
                }
            }

            // when all done, return the font to use
            return (font);
        }

    }
}
\end{Verbatim}

\section{UI Data binding}
\label{sec:UI-data-binding-WinForms}

Classically, data binding was used within applications to take advantage of data
stored in databases. Windows Forms data binding allows you to access data from
databases as well as data in other structures, such as arrays and collections
(assuming certain minimum requirements have been met). 

In Windows Forms, you can bind a UI control to a wide variety of structures,
from simple (arrays) to complex (data rows, data views, and so on). As a
minimum, a bindable structure must support the \verb!IList! interface.


 To do a simple data binding, e.g. a column value from a dataset table is
displayed in the Windows Form control, we first need to know which data
source we want to connect
\begin{enumerate}
  \item database: MS SQL Server ( .xsd file - a typed dataset)
  
  \item web service: data and method of a WCF service, WCF Data Services or a
  Web service (the type of the returned objects can be varied)
  
  \item objects: work with data in existing objects (the object can be located
  in the project, or you need a reference (add a reference) to the selected
  object first before it shows up in the wizard)
  
  \item SharePoint: data from a SharePoint site
  
\end{enumerate}

A connection object need to be created for the UI form or
component. In Visual Studio, it is typically created as the result of using a
data wizard (Data Source Configuration Wizard)
\footnote{\url{http://msdn.microsoft.com/en-us/library/w4dd7z6t.aspx}} or
dragging data objects onto the form.

Run Data Source Configuration Wizard using either
\begin{enumerate}
  \item Project / Add New Data Source
  \item Data Source Window choose Add New Data Source
  \item From the property of the bindable controls, select Add New Data Source
  command
\end{enumerate}



Once you added the database/data source, you choose {\bf Database Model} to
generate a dataset or EDM (Entity Data Model)
\begin{enumerate}
  \item ADO.NET: \url{http://msdn.microsoft.com/en-us/library/aa984058.aspx}
   \item ASP.NET : \url{http://msdn.microsoft.com/en-us/library/6759sth4.aspx}
\end{enumerate}

Next, choose your data connection
\begin{itemize}
  \item Choose an existing connection (from the list of connections) or create a
  new one (click {\it New Connection})
  \item Set values in {\it Connection Properties} dialog box
  \item read-only information is displayed in {\it Connection Properties} dialog
  box
\end{itemize}

Finally, save the {\bf Connection String} to the App. Configuration file. Type a
name for the connection or use the provided default name. If the database
connection changes, you can modify the connection string in the application
configuration file instead of editing the source code and recompiling your
application. How to edit a connection string
\url{http://msdn.microsoft.com/en-us/library/ms171887.aspx}

Look at {\bf Data Sources} window, once the datasource is available,
\footnote{\url{http://msdn.microsoft.com/en-us/library/6ckyxa83.aspx}} then you
can create data-bound controls (i.e. a control that bind to data) by dragging
the data source to a design surface
\url{http://msdn.microsoft.com/en-us/library/w4dd7z6t.aspx}

\begin{mdframed}
In some situations, it's convenient to create a connection object without the
assistance of any data design tools. E.g.: to create an ADO.NET
programmatically, see
\url{http://msdn.microsoft.com/en-us/library/32c5dh3b.aspx}
\end{mdframed}

  


\url{http://msdn.microsoft.com/en-us/library/aa984214.aspx}


\section{Text rendering: GDI vs. GDI+}
\label{sec:text_rendering}

Some new UI controls use GDI based \verb!TextRenderer! class to do text
rendering; the old one use GDI+ based \verb!Graphics! class. This is the change
made from .NET Framework 2.0 to resolve problems in performance
and localization in GDI+. GDI calculates character spacing and word wrapping
differently from GDI+, which can cause text for controls that use
\verb!TextRenderer! looks different from text rendered by \verb!Graphics! class.
To resolve this incompatibility, you can set the UseCompatibleTextRendering
property to true.  

In most cases, if your application is not being upgraded from .NET Framework
1.0 or .NET Framework 1.1, it is recommended that you leave
UseCompatibleTextRendering set to the default value of false. 

We use
\begin{verbatim}
using Systems.Windows.Forms;

Application.EnableVisualStyles();
 // to use GDI based TextRenderer class for text rendering
Applications.SetCompatibleTextRenderingDefault(false);


Application.Run(new Form1());
\end{verbatim}
See Sect.\ref{sec:Winforms.Application}

IMPORTANT: You should never call this method if your Windows Forms code is
hosted in another application, such as Internet Explorer. Only call this method
in stand-alone Windows Forms applications. 
\url{http://msdn.microsoft.com/en-us/library/system.windows.forms.application.setcompatibletextrenderingdefault(v=vs.110).aspx}